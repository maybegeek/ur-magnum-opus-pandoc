\documentclass[12pt,ngerman,a4paper,DIV=11]{scrartcl}
\usepackage{lmodern}
\usepackage{setspace}
\setstretch{1.7}
\usepackage{amssymb,amsmath}
\usepackage{ifxetex,ifluatex}
\usepackage{fixltx2e}
\usepackage{mathspec}
\usepackage{xltxtra,xunicode}

\usepackage{fontspec}

\defaultfontfeatures{Mapping=tex-text,Scale=MatchLowercase}

\newcommand{\euro}{€}

\setmainfont[%
%  RawFeature=-calt,%disable Q.long + contextual alternatives
  Path=./Template/Fonts/EBGaramond/otf/,%
  UprightFont=EBGaramond12-Regular.otf,%
  BoldFont=EBGaramond12-Regular.otf,%
  BoldFeatures={FakeBold=1.8},%
  ItalicFont=EBGaramond12-Italic.otf,%
  BoldItalicFont=EBGaramond12-Italic.otf,%
  BoldItalicFeatures={FakeBold=1.8},%
  SmallCapsFont=EBGaramondSC12-Regular.otf%
]{EBGaramond}

\setsansfont[%
%  RawFeature=-calt,%disable Q.long + contextual alternatives 
  Path=./Template/Fonts/EBGaramond/otf/,%
  UprightFont=EBGaramond12-Regular.otf,%
  BoldFont=EBGaramond12-Regular.otf,%
  BoldFeatures={FakeBold=1.8},%
  ItalicFont=EBGaramond12-Italic.otf,%
  BoldItalicFont=EBGaramond12-Italic.otf,%
  BoldItalicFeatures={FakeBold=1.8},%
  SmallCapsFont=EBGaramondSC12-Regular.otf%
]{EBGaramond}

%\setsansfont[%
%  Path=Template/Fonts/FiraSans/,%
%  UprightFont=*-Regular,%
%  ItalicFont=*-Italic,%
%  BoldFont=*-Bold,%
%  BoldItalicFont=*-BoldItalic%
%]{FiraSans}

\setmonofont[%
  Path=Template/Fonts/FiraMono/,%
  UprightFont=*-Regular,%
  BoldFont=*-Bold,%
  Mapping=tex-ansi,
]{FiraMono}

%\setmonofont[Mapping=tex-ansi]{Droid Sans Mono}

\setmathfont(Digits,Latin,Greek)[Path=Template/Fonts/TexGyrePagellaMath/otf/,UprightFont=texgyrepagella-math]{texgyrepagella-math}

% use upquote if available, for straight quotes in verbatim environments
\IfFileExists{upquote.sty}{\usepackage{upquote}}{}
% use microtype if available
\IfFileExists{microtype.sty}{%
\usepackage{microtype}
\UseMicrotypeSet[protrusion]{basicmath} % disable protrusion for tt fonts
}{}


\usepackage[bibstyle=%
fiwi2,%
citestyle=fiwi2,%
dashed=false,%
backend=biber,%
origyearwithyear=true,%
publisher=true,%
labeldate=true,%
pages=true,%
series=true,%
ibidtracker=false,%
filmruntime=true,%
citefilm=full%
]{biblatex}
\bibliography{Quellen/Quellen}


\usepackage{longtable,booktabs}
\usepackage{graphicx}
\makeatletter
\def\maxwidth{\ifdim\Gin@nat@width>\linewidth\linewidth\else\Gin@nat@width\fi}
\def\maxheight{\ifdim\Gin@nat@height>\textheight\textheight\else\Gin@nat@height\fi}
\makeatother
% Scale images if necessary, so that they will not overflow the page
% margins by default, and it is still possible to overwrite the defaults
% using explicit options in \includegraphics[width, height, ...]{}
\setkeys{Gin}{width=\maxwidth,height=\maxheight,keepaspectratio}
\ifxetex
  \usepackage[setpagesize=false, % page size defined by xetex
              unicode=false, % unicode breaks when used with xetex
              xetex]{hyperref}
\else
  \usepackage[unicode=true]{hyperref}
\fi
\hypersetup{breaklinks=true,
            bookmarks=true,
            pdfauthor={Hans Dampf},
            pdftitle={Dinge und Sachen},
            colorlinks=true,
            citecolor=black,
            urlcolor=black,
            linkcolor=black,
            pdfborder={0 0 0}}
\urlstyle{same}  % don't use monospace font for urls
\setlength{\parindent}{0pt}
\setlength{\parskip}{6pt plus 2pt minus 1pt}
\setlength{\emergencystretch}{3em}  % prevent overfull lines
\setcounter{secnumdepth}{5}


\usepackage[ngerman]{babel}
\usepackage{polyglossia}
\setdefaultlanguage[spelling=new]{german}
%\usepackage[autostyle,german=guillemets]{csquotes}
\usepackage[autostyle,german=quotes]{csquotes}
\MakeOuterQuote{"}

\title{Dinge und Sachen\\\vspace{0.5em}{\large Eine exhaustive Studie zur Lage der Dinge im Wandel der Zeiten}}
\author{Hans Dampf}
%\date{}
%
\renewcommand*{\newunitpunct}{\adddot\space}
\renewcommand*{\titlepagestyle}{empty}
%
\makeatletter
\g@addto@macro\tableofcontents{\clearpage}
\g@addto@macro\listoftables{\thispagestyle{plain}}
\g@addto@macro\listoffigures{\thispagestyle{plain}}
\makeatother
%
\deffootnote[2em]{2em}{2.2em}{\thefootnotemark\ }
%
\usepackage{varioref}
% 
\renewcommand*{\sectionmarkformat}{}
%
\makeatletter
\newif \if@mainmatter \@mainmattertrue
\newcommand*\frontmatter{\clearpage\thispagestyle{plain}\@mainmatterfalse\pagenumbering{roman}}
\newcommand*\mainmatter{\clearpage\thispagestyle{plain}\@mainmattertrue\pagenumbering{arabic}}
\newcommand*\backmatter{\clearpage\thispagestyle{plain}\@mainmatterfalse}
\makeatother
%
%
\renewbibmacro*{filmtitle}
{\iffieldundef{maintitle}
{\printtext{\printfield[film]{title}}}
{\printfield[film]{maintitle}\newunit}%
\iffieldundef{subtitle}%
{}%
{\setunit{}%
\printtext{\addspace\printfield{subtitle}}}%
\iffieldundef{volume}%
{}%
{\printfield[season]{volume}}%
\iffieldundef{number}%
{}%
{\addcomma\addspace\printfield[episode]{number}}%
\iffieldundef{maintitle}%
{}%
{\addcolon\addspace\printfield[film]{title}}%
\addspace\mkbibparens{\printfield{year}\iffieldundef{origyear}{}{\printtext{/}\printorigdate}}%
\ifpunctmark{!}{\unspace .\newunit}{\addcolon}}%
%
%
\DeclareBibliographyDriver{movie}{%
  \usebibmacro{bibindex}%
  \usebibmacro{begentry}%
  \newblock%
  \usebibmacro{filmtitle}%
  \newunit\newblock%
  \usebibmacro{movie:creators}%
  \iffieldundef{entrysubtype}
	{}%
	{\iffieldequalstr{entrysubtype}{serial}%
		{\usebibmacro{movie:serials}}%
		{\iffieldequalstr{entrysubtype}{tv}%
			{\usebibmacro{movie:tv}}%
			{\usebibmacro{movie:regular}}}}%
  \iffieldundef{pagetotal}
  	{}
  	{%
 	\iftoggle{filmruntime}%
 		{\adddot\addspace\printfield{pagetotal}}%
 		{}}
 \iffieldundef{note}%
 	{}
 	{\printfield{note}}%
  \newunit\newblock
  \iftoggle{bbx:isbn}
    {\printfield{isan}}
    {}%
 \newunit\newblock
%\usebibmacro{doi+eprint+url}
 \usebibmacro{pageref}
 \iflistundef{location}%
 {}%
 {\printlist{location}\printtext{: }}
 \iffieldundef{howpublished}%
 {}%
 {\printfield{howpublished}}
 \newunit\newblock
 \usebibmacro{url+urldate}
\usebibmacro{finentry}}
%
%
\DeclareBibliographyDriver{book}{%
  \usebibmacro{bibindex}%
  \usebibmacro{begentry}%
  \iftoggle{dontprintorig}
  {}
  {\ifnameundef{author}%
  {\ifnameundef{editor}%
  {}
  {\usebibmacro{editor}\addspace}}%
  {\usebibmacro{author/translator+others}}%
  \usebibmacro{date+extrayear}}%
  \newblock
  \usebibmacro{mtitle+mstitle+vol+part+title+stitle}%
  \newunit\newblock
  \ifnameundef{author}
  	{}
	{\usebibmacro{byeditor+others}}%
  \newunit\newblock
  \printfield{note}%
  \newunit
  \printfield{volumes}%
  \newunit\newblock
  \usebibmacro{ser+num}%
  \newunit\newblock
  \printfield{edition}%
  \newunit\newblock%
  \usebibmacro{publ+loc+origyear}%
  \usebibmacro{chap+pag}%
  \newblock%\newunit
 \iffieldundef{howpublished}%
 {}%
 {\printfield{howpublished}\adddot}
  \usebibmacro{doi+eprint+url}%
  \addspace\usebibmacro{related}%
  \newunit\newblock
  \iftoggle{bbx:isbn}
    {\printfield{isbn}}
    {}%
  \newblock
  \usebibmacro{addendum+pubstate}%
  \newunit\newblock
  \usebibmacro{pageref}%
\finentry}
%
%
\renewbibmacro*{director:first-last}[4]{%
  \usebibmacro{name:delim}{#1#2#3}%
  \usebibmacro{name:hook}{#1#2#3}%
  \ifblank{#1}{}{\mkbibnamelast{#1}\isdot\addlowpenspace\addcomma\addspace}%
  \ifblank{#3}{}{%
    \mkbibnameprefix{#3}\isdot
    \ifpunctmark{'}
      {}
      {\ifuseprefix{\addhighpenspace}{\addlowpenspace}}}%
  \mkbibnamefirst{#2}\isdot
  \ifblank{#4}{}{\addlowpenspace\mkbibnameaffix{#4}\isdot}}
%
%
\renewbibmacro*{cite}{%
  \global\boolfalse{cbx:loccit}%
  \iffieldundef{shorthand}
    {\ifthenelse{\ifciteibid\AND\NOT\iffirstonpage}
       {\usebibmacro{cite:ibid}}
       {\ifthenelse{\ifnameundef{labelname}\OR\iffieldundef{labelyear}}
          %{\usebibmacro{cite:label}%
          {\printtext[bibhyperref]{\iffieldundef{shorttitle}{\printfield[film]{title}}{\printfield[film]{shorttitle}}}%
           \setunit{\addspace}}
          {\printnames{labelname}%
           \setunit{\nameyeardelim}}%
        \printfield{year}\iffieldundef{origyear}{}{\printtext{/}\printorigdate}}}%
    {\usebibmacro{cite:shorthand}}}    
%
%
\renewbibmacro*{textcite}{%
  \global\boolfalse{cbx:loccit}%
  \iffieldundef{type}%
  {%
  \ifnameundef{labelname}
    {\iffieldundef{shorthand}
       {\usebibmacro{cite:label}%
        \setunit{%
          \global\booltrue{cbx:parens}%
          \addspace\bibopenparen}%
        \ifnumequal{\value{citecount}}{1}
          {\usebibmacro{prenote}}
          {}%
        \usebibmacro{cite:labelyear+extrayear}}
       {\usebibmacro{cite:shorthand}}}
    {\printnames{labelname}%
     \setunit{%
       \global\booltrue{cbx:parens}%
       \addspace\bibopenparen}%
     \ifnumequal{\value{citecount}}{1}
       {\usebibmacro{prenote}}
       {}%
     \iffieldundef{shorthand}
       {\ifthenelse{\ifciteibid\AND\NOT\iffirstonpage}
          {\usebibmacro{cite:ibid}}
          {\iffieldundef{labelyear}
             {\usebibmacro{cite:label}}
             {\usebibmacro{cite:labelyear+extrayear}}}}
       {\usebibmacro{cite:shorthand}}}}%
       {\printtext[bibhyperref]{\iffieldundef{shorttitle}{\printfield[film]{title}}{\printfield[film]{shorttitle}}\addspace\mkbibparens{\printfield{year}}}}%
}
%
%
\renewbibmacro*{cite:label}{%
  \ifnameundef{labelname}
    {\BibliographyWarning{Missing author/editor+year or label}}
    {%
    \iffieldundef{type}{%
    \printtext[bibhyperref]{\printnames{labelname}}}%
    {\printfield{year}\iffieldundef{origyear}{}{\printtext{/}\printorigdate}}%
    }}
%
%
\renewcommand{\mkbibnamelast}[1]{\textsc{#1}}


\begin{document}

\frontmatter
\begin{titlepage}
  \begin{center}

  \hbox{\hspace{2.74cm}\includegraphics[width=0.45\textwidth]{Template/UR-Logo/ur-logo-mit-text.pdf}}

  \vspace{1.8cm}

  {\huge\textbf{Dinge und Sachen}}

    \vspace{0.8cm}
  {\large{Eine exhaustive Studie zur Lage der Dinge im Wandel der Zeiten}}
  
  \vspace{1.5cm}

    Bachelorarbeit im Fach Superwissenschaft
  
    Institut für Information und Medien, Sprache und Kultur (I:IMSK)
  
  \vspace{0.8cm}
  \begin{center}
  \begin{tabular}{ r c l }
  von:             &  & Hans Dampf                                     \\
  Adresse:         &  & \parbox[t]{4cm}{ Dingsheimer Straße 12 \\ 12345 Hallodrihausen }  \\
                   &  &                                              \\
  Matrikelnummer:  &  & 123 456 7                                \\
                   &  &                                              \\
  Erstgutachter:   &  & Prof. Dr. Christian Wolff                                 \\
  Zweitgutachter:  &  & Prof. Dr. Rainer Hammwöhner                                \\
                   &  &                                              \\
  Abgabedatum:     &  & 21.\,Mai\,2015
  \end{tabular}
  \end{center}
  \end{center}
\end{titlepage}

\begin{titlepage}
  \begin{center}

  \hbox{\hspace{2.74cm}\includegraphics[width=0.45\textwidth]{Template/UR-Logo/ur-logo-mit-text.pdf}}

  \vspace{1.8cm}

  {\huge\textbf{Dinge und Sachen}}

    \vspace{0.8cm}
  {\large{Eine exhaustive Studie zur Lage der Dinge im Wandel der Zeiten}}
  
  \vspace{1.5cm}

    Hausarbeit im Kurs: "Digitale Welten"
  
    (Prof. Dr. Hans Wichtig)
  
  \vspace{0.8cm}

  \begin{center}
  \begin{tabular}{ r c l }
  von:               &  & Hans Dampf                                     \\
  Matrikelnummer:    &  & 123 456 7                                \\
                     &  &                                              \\
  Semester:          &  & Sommersemester 2014                                \\
  Fächerkombination: &  & \parbox[t]{6cm}{Medienwissenschaft (Informationswissenschaft/Medieninformatik)}                \\
  Modul:             &  & M-05                                   \\
  Fachsemester:      &  & 3                            \\
                     &  &                                              \\
  Abgabedatum:       &  & 21.\,Mai\,2015
  \end{tabular}
  \end{center}
  \end{center}
\end{titlepage}


{
\hypersetup{linkcolor=black}
\setcounter{tocdepth}{3}
\thispagestyle{plain}
\tableofcontents
}
\listoftables
\listoffigures

\mainmatter
\section{Hier geht es los}\label{hier-geht-es-los}

Er hörte leise Schritte hinter sich. Das bedeutete nichts Gutes. Wer
würde ihm schon folgen, spät in der Nacht und dazu noch in dieser engen
Gasse mitten im übel beleumundeten Hafenviertel? Gerade jetzt, wo er das
Ding seines Lebens gedreht hatte und mit der Beute verschwinden wollte!

alles in runden Klammern

\begin{itemize}
\itemsep1pt\parskip0pt\parsep0pt
\item
  \autocite{dotzler:2008}\\ \texttt{{[}@dotzler:2008{]}}
\item
  \autocite[9-12]{dotzler:2008}\\ \texttt{{[}@dotzler:2008, 9-12{]}}
\item
  \autocites[9-12]{dotzler:2008}[außerdem][7-13]{dotzler:1995}\\
  \texttt{{[}@dotzler:2008, 9-12; außerdem @dotzler:1995, 7-13{]}}
\end{itemize}

Unterdrückung des Autors im Klammernausdruck.

\begin{itemize}
\itemsep1pt\parskip0pt\parsep0pt
\item
  \autocite*{dotzler:2008}\\ \texttt{{[}-@dotzler:2008{]}}
\item
  \autocite*[9-12]{dotzler:2008}\\ \texttt{{[}-@dotzler:2008, 9-12{]}}
\item
  \autocites[9-12]{dotzler:2008}[außerdem][7-13]{dotzler:1995}\\
  \texttt{{[}-@dotzler:2008, 9-12; außerdem -@dotzler:1995, 7-13{]}}
\end{itemize}

Autor vor die Klammern ziehen

\begin{itemize}
\itemsep1pt\parskip0pt\parsep0pt
\item
  \textcite{dotzler:2008}\\ \texttt{@dotzler:2008}
\item
  \textcite[9-12]{dotzler:2008}\\ \texttt{@dotzler:2008 {[}9-12{]}}
\item
  \textcite[9-12]{dotzler:2008} außerdem \textcite[7-13]{dotzler:1995}\\
  \texttt{@dotzler:2008 {[}9-12{]}; außerdem @dotzler:1995 {[}7-13{]}}
\end{itemize}

Hatte einer seiner zahllosen Kollegen dieselbe Idee gehabt, ihn
beobachtet und abgewartet, um ihn nun um die Früchte seiner Arbeit zu
erleichtern? Oder gehörten die Schritte hinter ihm zu einem der
unzähligen Gesetzeshüter dieser Stadt, und die stählerne Acht um seine
Handgelenke würde gleich zuschnappen?

Er konnte die Aufforderung stehen zu bleiben schon hören. Gehetzt sah er
sich um. Plötzlich erblickte er den schmalen Durchgang. Blitzartig
drehte er sich nach rechts und verschwand zwischen den beiden Gebäuden.
Beinahe wäre er dabei über den umgestürzten Mülleimer gefallen, der
mitten im Weg lag.

Er versuchte, sich in der Dunkelheit seinen Weg zu ertasten und
erstarrte: Anscheinend gab es keinen anderen Ausweg aus diesem kleinen
Hof als den Durchgang, durch den er gekommen war. Die Schritte wurden
lauter und lauter, er sah eine dunkle Gestalt um die Ecke biegen.

Fieberhaft irrten seine Augen durch die nächtliche Dunkelheit und
suchten einen Ausweg. War jetzt wirklich alles vorbei, waren alle Mühe
und alle Vorbereitungen umsonst?

Er konnte die Aufforderung stehen zu bleiben schon hören. Gehetzt sah er
sich um. Plötzlich erblickte er den schmalen Durchgang. Blitzartig
drehte er sich nach rechts und verschwand zwischen den beiden Gebäuden.
Beinahe wäre er dabei über den umgestürzten Mülleimer gefallen, der
mitten im Weg lag.

Er versuchte, sich in der Dunkelheit seinen Weg zu ertasten und
erstarrte: Anscheinend gab es keinen anderen Ausweg aus diesem kleinen
Hof als den Durchgang, durch den er gekommen war. Die Schritte wurden
lauter und lauter, er sah eine dunkle Gestalt um die Ecke biegen.

Fieberhaft irrten seine Augen durch die nächtliche Dunkelheit und
suchten einen Ausweg. War jetzt wirklich alles vorbei, waren alle Mühe
und alle Vorbereitungen umsonst?

\section{Ein weiterer, wichtiger
Punkt}\label{ein-weiterer-wichtiger-punkt}

Er hörte leise Schritte hinter sich. Das bedeutete nichts Gutes. Wer
würde ihm schon folgen, spät in der Nacht und dazu noch in dieser engen
Gasse mitten im übel beleumundeten Hafenviertel? Gerade jetzt, wo er das
Ding seines Lebens gedreht hatte und mit der Beute verschwinden wollte!

Hatte einer seiner zahllosen Kollegen dieselbe Idee gehabt, ihn
beobachtet und abgewartet, um ihn nun um die Früchte seiner Arbeit zu
erleichtern? Oder gehörten die Schritte hinter ihm zu einem der
unzähligen Gesetzeshüter dieser Stadt, und die stählerne Acht um seine
Handgelenke würde gleich zuschnappen?

\subsection{noch tiefer ins Detail}\label{noch-tiefer-ins-detail}

Er konnte die Aufforderung stehen zu bleiben schon hören. Gehetzt sah er
sich um. Plötzlich erblickte er den schmalen Durchgang. Blitzartig
drehte er sich nach rechts und verschwand zwischen den beiden Gebäuden.
Beinahe wäre er dabei über den umgestürzten Mülleimer gefallen, der
mitten im Weg lag.

Er versuchte, sich in der Dunkelheit seinen Weg zu ertasten und
erstarrte: Anscheinend gab es keinen anderen Ausweg aus diesem kleinen
Hof als den Durchgang, durch den er gekommen war. Die Schritte wurden
lauter und lauter, er sah eine dunkle Gestalt um die Ecke biegen.

Fieberhaft irrten seine Augen durch die nächtliche Dunkelheit und
suchten einen Ausweg. War jetzt wirklich alles vorbei, waren alle Mühe
und alle Vorbereitungen umsonst?

Er konnte die Aufforderung stehen zu bleiben schon hören. Gehetzt sah er
sich um. Plötzlich erblickte er den schmalen Durchgang. Blitzartig
drehte er sich nach rechts und verschwand zwischen den beiden Gebäuden.
Beinahe wäre er dabei über den umgestürzten Mülleimer gefallen, der
mitten im Weg lag.

\subsection{und ebenfalls noch
wichtig}\label{und-ebenfalls-noch-wichtig}

Er hörte leise Schritte hinter sich. Das bedeutete nichts Gutes. Wer
würde ihm schon folgen, spät in der Nacht und dazu noch in dieser engen
Gasse mitten im übel beleumundeten Hafenviertel? Gerade jetzt, wo er das
Ding seines Lebens gedreht hatte und mit der Beute verschwinden wollte!

Hatte einer seiner zahllosen Kollegen dieselbe Idee gehabt, ihn
beobachtet und abgewartet, um ihn nun um die Früchte seiner Arbeit zu
erleichtern? Oder gehörten die Schritte hinter ihm zu einem der
unzähligen Gesetzeshüter dieser Stadt, und die stählerne Acht um seine
Handgelenke würde gleich zuschnappen?

Er konnte die Aufforderung stehen zu bleiben schon hören. Gehetzt sah er
sich um. Plötzlich erblickte er den schmalen Durchgang. Blitzartig
drehte er sich nach rechts und verschwand zwischen den beiden Gebäuden.
Beinahe wäre er dabei über den umgestürzten Mülleimer gefallen, der
mitten im Weg lag.

\begin{longtable}[c]{@{}ll@{}}
\caption{Titelblattangaben zum jeweiligen Dokumenttyp.}\tabularnewline
\toprule
Seminararbeit & Abschlussarbeit\tabularnewline
\midrule
\endfirsthead
\toprule
Seminararbeit & Abschlussarbeit\tabularnewline
\midrule
\endhead
Semester & Titel der Arbeit\tabularnewline
Lehrveranstaltung & Universität\tabularnewline
Dozent & Fakultät\tabularnewline
Modul des Leistungsnachweises & Lehrstuhl\tabularnewline
Verfasser (Name, Matrikelnr.) & Verfasser (Name, Anschrift,
Matrikelnr.)\tabularnewline
Titel der Arbeit & Erst- und Zweitgutachter\tabularnewline
Abgabedatum & Abgabedatum\tabularnewline
\bottomrule
\end{longtable}

Er versuchte, sich in der Dunkelheit seinen Weg zu ertasten und
erstarrte: Anscheinend gab es keinen anderen Ausweg aus diesem kleinen
Hof als den Durchgang, durch den er gekommen war. Die Schritte wurden
lauter und lauter, er sah eine dunkle Gestalt um die Ecke biegen.

Fieberhaft irrten seine Augen durch die nächtliche Dunkelheit und
suchten einen Ausweg. War jetzt wirklich alles vorbei, waren alle Mühe
und alle Vorbereitungen umsonst?

Er konnte die Aufforderung stehen zu bleiben schon hören. Gehetzt sah er
sich um. Plötzlich erblickte er den schmalen Durchgang. Blitzartig
drehte er sich nach rechts und verschwand zwischen den beiden Gebäuden.
Beinahe wäre er dabei über den umgestürzten Mülleimer gefallen, der
mitten im Weg lag.

\section{Bevor zum Anschluss}\label{bevor-zum-anschluss}

Er hörte leise Schritte hinter sich. Das bedeutete nichts Gutes. Wer
würde ihm schon folgen, spät in der Nacht und dazu noch in dieser engen
Gasse mitten im übel beleumundeten Hafenviertel? Gerade jetzt, wo er das
Ding seines Lebens gedreht hatte und mit der Beute verschwinden wollte!

Hatte einer seiner zahllosen Kollegen dieselbe Idee gehabt, ihn
beobachtet und abgewartet, um ihn nun um die Früchte seiner Arbeit zu
erleichtern? Oder gehörten die Schritte hinter ihm zu einem der
unzähligen Gesetzeshüter dieser Stadt, und die stählerne Acht um seine
Handgelenke würde gleich zuschnappen?

Er konnte die Aufforderung stehen zu bleiben schon hören. Gehetzt sah er
sich um. Plötzlich erblickte er den schmalen Durchgang. Blitzartig
drehte er sich nach rechts und verschwand zwischen den beiden Gebäuden.
Beinahe wäre er dabei über den umgestürzten Mülleimer gefallen, der
mitten im Weg lag.

Er versuchte, sich in der Dunkelheit seinen Weg zu ertasten und
erstarrte: Anscheinend gab es keinen anderen Ausweg aus diesem kleinen
Hof als den Durchgang, durch den er gekommen war. Die Schritte wurden
lauter und lauter, er sah eine dunkle Gestalt um die Ecke biegen.

Fieberhaft irrten seine Augen durch die nächtliche Dunkelheit und
suchten einen Ausweg. War jetzt wirklich alles vorbei, waren alle Mühe
und alle Vorbereitungen umsonst?

Er konnte die Aufforderung stehen zu bleiben schon hören. Gehetzt sah er
sich um. Plötzlich erblickte er den schmalen Durchgang. Blitzartig
drehte er sich nach rechts und verschwand zwischen den beiden Gebäuden.
Beinahe wäre er dabei über den umgestürzten Mülleimer gefallen, der
mitten im Weg lag.

\fullcite{reimann:2012}

\fullcite{kubrick:2013}

An der Zeit auch mal ein Bild zu verwenden.

\begin{figure}[htbp]
\centering
\includegraphics{Bilder/ur-logo.png}
\caption{Logo der Universität Regensburg (png)}
\end{figure}

\begin{figure}[htbp]
\centering
\includegraphics{Bilder/ur-logo.jpg}
\caption{Logo der Universität Regensburg (jpg)}
\end{figure}

\begin{figure}[htbp]
\centering
\includegraphics{Bilder/ur-logo.pdf}
\caption{Logo der Universität Regensburg (pdf)}
\end{figure}

\nocite{*}

\backmatter
\printbibheading[title={Quellen}]

\printbibliography[notkeyword=film,heading=subbibintoc,title={Literaturverzeichnis}]

\printbibliography[keyword=film,heading=subbibintoc,title={Filmverzeichnis}]


\clearpage
\thispagestyle{plain}
\markboth{}{}
\section*{Eidesstattliche Erklärung}
\addcontentsline{toc}{section}{Eidesstattliche Erklärung}


Ich habe die Arbeit selbständig verfasst, keine anderen als die angegebenen 
Quellen und Hilfsmittel benutzt und die Arbeit bisher keiner anderen
Prüfungsbehörde vorgelegt.

\vspace{1cm}

  \begin{center}
  \begin{tabular}{ r c l }
  (Hans Dampf)         &  & \dotfill                                     \\
                     &  & \parbox[t]{8cm}{\strut}                      \\
  (Ort und Datum)    &  & \dotfill                                     \\
  \end{tabular}
  \end{center}



\end{document}
